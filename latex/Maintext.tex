\thepage
\section{Introduction}
In recent years open source software solutions have become widely popular and frequently used in both scientific and enterprise use, which can be attributed to a number of factors, most importantly the ease of development and deployment of IT projects, improved cybersecurity and enhanced scalability \cite{pwcLeadingBenefitsOpensource2016}. This increases the contribution to open source projects from enterprises and individuals alike. Due to its nature, open source software projects are driven by community contributions, and depend heavily on active participation in all phases of the project. Because there is a high dependency on the community in open source software projects, by understanding how contributions are included and what patterns emerge we can gain valuable insight into the project's current state and its trajectory.


\subsection{Motivation of research problem and research question}

rhoncus ut, imperdiet a, venenatis vitae, justo. Nullam dictum felis eu pede mollis pretium. Integer tincidunt. Cras dapibus. Vivamus elementum semper nisi. Aenean vulputate eleifend tellus. Aenean leo ligula, porttitor eu, consequat vitae, eleifend ac, enim. Aliquam lorem ante, dapibus in, viverra quis, feugiat a, tellus.

% Summary of background literature and state of the art solutions
\subsection{Literature review}
rhoncus ut, imperdiet a, venenatis vitae, justo. Nullam dictum felis eu pede mollis pretium. Integer tincidunt. Cras dapibus. Vivamus elementum semper nisi. Aenean vulputate eleifend tellus. Aenean leo ligula, porttitor eu, consequat vitae, eleifend ac, enim. Aliquam lorem ante, dapibus in, viverra quis, feugiat a, tellus.

\section{Proposed research method}

rhoncus ut, imperdiet a, venenatis vitae, justo. Nullam dictum felis eu pede mollis pretium. Integer tincidunt. Cras dapibus. Vivamus elementum semper nisi. Aenean vulputate eleifend tellus. Aenean leo ligula, porttitor eu, consequat vitae, eleifend ac, enim. Aliquam lorem ante, dapibus in, viverra quis, feugiat a, tellus.rhoncus ut, imperdiet a, venenatis vitae, justo. Nullam dictum felis eu pede mollis pretium. Integer tincidunt. Cras dapibus. Vivamus elementum semper nisi. Aenean vulputate eleifend tellus. Aenean leo ligula, porttitor eu, consequat vitae, eleifend ac, enim. Aliquam lorem ante, dapibus in, viverra quis, feugiat a, tellus.

\section{Outline of thesis}

\subsection{Preliminary literature list}

\subsection{Work plan including milestones}

Table \ref{table1} is an example of a table.



\begin{table}[]
\centering
\caption{Example of a table}
\label{table1}
\begin{tabular}{lllll}
 1 & a & - & - & - \\
 2 & b & - & - & - \\
 3 & c & - & - & - \\
 4 & d & - & - &-
\end{tabular}
\end{table}
rhoncus ut, imperdiet a, venenatis vitae, justo. Nullam dictum felis eu pede mollis pretium. Integer tincidunt. Cras dapibus. Vivamus elementum semper nisi. Aenean vulputate eleifend tellus. Aenean leo ligula, porttitor eu, consequat vitae, eleifend ac, enim. Aliquam lorem ante, dapibus in, viverra quis, feugiat a, tellus \cite{sekaraFundamentalStructuresDynamic2016}.
\newline
\newline

``This is a direct quote.'' \cite{goteAnalysingTimeStampedCoEditing2019} 